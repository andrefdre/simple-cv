%%%%%%%%%%%%%%%%%
% This is an sample CV template created using altacv.cls
% (v1.1.4, 27 July 2018) written by LianTze Lim (liantze@gmail.com). Now compiles with pdfLaTeX, XeLaTeX and LuaLaTeX.
% 
%% It may be distributed and/or modified under the
%% conditions of the LaTeX Project Public License, either version 1.3
%% of this license or (at your option) any later version.
%% The latest version of this license is in
%%    http://www.latex-project.org/lppl.txt
%% and version 1.3 or later is part of all distributions of LaTeX
%% version 2003/12/01 or later.
%%%%%%%%%%%%%%%%
\newcommand{\RNum}[1]{\uppercase\expandafter{\romannumeral #1\relax}}
%% If you need to pass whatever options to xcolor
\PassOptionsToPackage{dvipsnames}{xcolor}

%% If you are using \orcid or academicons
%% icons, make sure you have the academicons 
%% option here, and compile with XeLaTeX
%% or LuaLaTeX.
% \documentclass[10pt,a4paper,academicons]{altacv}

%% Use the "normalphoto" option if you want a normal photo instead of cropped to a circle
% \documentclass[10pt,a4paper,normalphoto]{altacv}

\documentclass[10pt,a4paper]{altacv}
%% AltaCV uses the fontawesome and academicon fonts
%% and packages. 
%% See texdoc.net/pkg/fontawecome and http://texdoc.net/pkg/academicons for full list of symbols.
%% 
%% Compile with LuaLaTeX for best results. If you
%% want to use XeLaTeX, you may need to install
%% Academicons.ttf in your operating system's font 
%% folder.


% Change the page layout if you need to
\geometry{left=1cm,right=9cm,marginparwidth=6.8cm,marginparsep=1.2cm,top=1.25cm,bottom=1.25cm,footskip=2\baselineskip}

% Change the font if you want to.

% If using pdflatex:
\usepackage[T1]{fontenc}
\usepackage[utf8]{inputenc}
\usepackage[default]{lato}

% If using xelatex or lualatex:
% \setmainfont{Lato}

% Change the colours if you want to
\definecolor{Navy}{HTML}{000080}
\definecolor{SlateGrey}{HTML}{2E2E2E}
\definecolor{LightGrey}{HTML}{666666}
\colorlet{heading}{Navy}
\colorlet{accent}{Navy}
\colorlet{emphasis}{SlateGrey}
\colorlet{body}{LightGrey}

% Change the bullets for itemize and rating marker
% for \cvskill if you want to
\renewcommand{\itemmarker}{{\small\textbullet}}
\renewcommand{\ratingmarker}{\faCircle}
%% sample.bib contains your publications
\addbibresource{sample.bib}

\usepackage[colorlinks]{hyperref}

\begin{document}

\name{André Cardoso}
\tagline{Software Engineer}
% \photo{2.8cm}{Globe_High}
\personalinfo{%
  % Not all of these are required!
  % You can add your own with \printinfo{symbol}{detail}
  \email{andrefdre@gmail.com}
  \phone{(+351) 934391342}
%  \mailaddress{Address, Street, 00000 County}
 \location{Aveiro, Portugal}
%  \homepage{abhinavj004.github.io}
%  \twitter{@marissamayer}
  \linkedin{linkedin.com/in/andr\%C3\%A9-cardoso-8bb264223/}
  \newline
   \github{github.com/andrefdre} % I'm just making this up though.
  %% You MUST add the academicons option to \documentclass, then compile with LuaLaTeX or XeLaTeX, if you want to use \orcid or other academicons commands.
%   \orcid{orcid.org/0000-0000-0000-0000}
}

%% Make the header extend all the way to the right, if you want. 
\begin{fullwidth}
\makecvheader
\end{fullwidth}

%% Depending on your tastes, you may want to make fonts of itemize environments slightly smaller
% \AtBeginEnvironment{itemize}{\small}


%% Provide the file name containing the sidebar contents as an optional parameter to \cvsection.
%% You can always just use \marginpar{...} if you do
%% not need to align the top of the contents to any
%% \cvsection title in the "main" bar.
\cvsection[page1sidebar]{Experience}

\cvevent{Research Fellow / Dissertation}{Laboratory of Automation and Robotics | University of Aveiro}{February 2023 - Present}{}
\begin{itemize}
\item Implementation of a sophisticated \textbf{sunlight variation algorithm} in a data-driven simulator.  
\item  Integration of a cutting-edge \textbf{object variation algorithm} into the data-driven simulator.
\item By incorporating sophisticated techniques for \textbf{scene augmentation}, we were able to significantly enhance the ability of our models to generalize to previously unseen data.
\end{itemize}
\medskip
\divider
\cvevent{Machine Learning / Eletrical Engineer}{Automec - University of Aveiro}{September 2021 - Present}{}
\begin{itemize}
\item Implementation of multiple \textbf{neural networks} utilizing the \textbf{PyTorch} framework to control a car in the field of autonomous vehicle technology.
\item Development and fabrication of a high-quality \textbf{PCB} equipped with comprehensive power distribution and low-level control mechanisms.
\end{itemize}



\cvsection{Projects}

\cvevent{Dora the Mug Finder}{Pytorch / ROS / Open3D}{December 2022 -- February 2023}{}
\begin{itemize}
\item The utilization of advanced \textbf{point cloud processing} techniques to extract objects from complex three-dimensional scenes.
\item Employing a state-of-the-art deep \textbf{neural network} for object classification in the scene camera.
\item Utilizing \textbf{ROS} as the underlying framework for project development facilitated seamless collaboration and interoperability.
\end{itemize}
\medskip
\divider

\cvevent{Full-stack Website development}{React / NodeJS / MongoDB}{April 2022 -- July 2022}{}
\begin{itemize}
    \item The development of a \textbf{full-stack web application}  for a shopping website showcased our expertise in crafting robust and scalable software solutions.
\end{itemize}
\divider
\medskip

\cvevent{GPS guided RC car}{3D Printing / 3D Modelling / Arduino}{July 2019 -- Present}{}
\begin{itemize}
    \item Development of the \textbf{mechanical project} as well as its fabrication using \textbf{3D printing} technology.
    \item Development of the Software and Electrical project.
\end{itemize}
\divider
%\medskip

% Adapted from @Jake's answer from http://tex.stackexchange.com/a/82729/226
% \wheelchart{outer radius}{inner radius}{
% comma-separated list of value/text width/color/detail}


\clearpage




%% If the NEXT page doesn't start with a \cvsection but you'd
%% still like to add a sidebar, then use this command on THIS
%% page to add it. The optional argument lets you pull up the 
%% sidebar a bit so that it looks aligned with the top of the
%% main column.
% \addnextpagesidebar[-1ex]{page3sidebar}

\end{document}
