%%%%%%%%%%%%%%%%%
% This is an sample CV template created using altacv.cls
% (v1.1.4, 27 July 2018) written by LianTze Lim (liantze@gmail.com). Now compiles with pdfLaTeX, XeLaTeX and LuaLaTeX.
% 
%% It may be distributed and/or modified under the
%% conditions of the LaTeX Project Public License, either version 1.3
%% of this license or (at your option) any later version.
%% The latest version of this license is in
%%    http://www.latex-project.org/lppl.txt
%% and version 1.3 or later is part of all distributions of LaTeX
%% version 2003/12/01 or later.
%%%%%%%%%%%%%%%%
\newcommand{\RNum}[1]{\uppercase\expandafter{\romannumeral #1\relax}}
%% If you need to pass whatever options to xcolor
\PassOptionsToPackage{dvipsnames}{xcolor}

%% If you are using \orcid or academicons
%% icons, make sure you have the academicons 
%% option here, and compile with XeLaTeX
%% or LuaLaTeX.
% \documentclass[10pt,a4paper,academicons]{altacv}

%% Use the "normalphoto" option if you want a normal photo instead of cropped to a circle
% \documentclass[10pt,a4paper,normalphoto]{altacv}

\documentclass[10pt,a4paper]{altacv}
%% AltaCV uses the fontawesome and academicon fonts
%% and packages. 
%% See texdoc.net/pkg/fontawecome and http://texdoc.net/pkg/academicons for full list of symbols.
%% 
%% Compile with LuaLaTeX for best results. If you
%% want to use XeLaTeX, you may need to install
%% Academicons.ttf in your operating system's font 
%% folder.


% Change the page layout if you need to
\geometry{left=1cm,right=9cm,marginparwidth=6.8cm,marginparsep=1.2cm,top=1.25cm,bottom=1.25cm,footskip=2\baselineskip}

% Change the font if you want to.

% If using pdflatex:
\usepackage[T1]{fontenc}
\usepackage[utf8]{inputenc}
\usepackage[default]{lato}

% If using xelatex or lualatex:
% \setmainfont{Lato}

% Change the colours if you want to
\definecolor{Navy}{HTML}{000080}
\definecolor{SlateGrey}{HTML}{2E2E2E}
\definecolor{LightGrey}{HTML}{666666}
\colorlet{heading}{Navy}
\colorlet{accent}{Navy}
\colorlet{emphasis}{SlateGrey}
\colorlet{body}{LightGrey}

% Change the bullets for itemize and rating marker
% for \cvskill if you want to
\renewcommand{\itemmarker}{{\small\textbullet}}
\renewcommand{\ratingmarker}{\faCircle}
%% sample.bib contains your publications
\addbibresource{sample.bib}

\usepackage[colorlinks]{hyperref}

\begin{document}

\name{André Cardoso}
\tagline{Engenheiro Mecânico}
\photo{2.8cm}{logo}
\personalinfo{%
  % Not all of these are required!
  % You can add your own with \printinfo{symbol}{detail}
  \email{andrefdre@gmail.com}
  \phone{(+351) 934391342}
%  \mailaddress{Address, Street, 00000 County}
 \location{Aveiro, Portugal}
%  \homepage{abhinavj004.github.io}
%  \twitter{@marissamayer}
  \linkedin{linkedin.com/in/andre-cardoso-pt/}
  \newline
   \github{github.com/andrefdre} % I'm just making this up though.
  %% You MUST add the academicons option to \documentclass, then compile with LuaLaTeX or XeLaTeX, if you want to use \orcid or other academicons commands.
%   \orcid{orcid.org/0000-0000-0000-0000}
}

%% Make the header extend all the way to the right, if you want. 
\begin{fullwidth}
\makecvheader
\end{fullwidth}

%% Depending on your tastes, you may want to make fonts of itemize environments slightly smaller
% \AtBeginEnvironment{itemize}{\small}


%% Provide the file name containing the sidebar contents as an optional parameter to \cvsection.
%% You can always just use \marginpar{...} if you do
%% not need to align the top of the contents to any
%% \cvsection title in the "main" bar.
\cvsection[page1sidebar]{Experiência}

\cvevent{Emprego a Tempo Inteiro}{Volper}{Setembro 2023 - Presente}{}
\begin{itemize}
\item Desenvolvi e implementei programas em \textbf{PLC} para automação industrial, otimizando o desempenho e a eficiência das máquinas.  
\item Liderei projetos de \textbf{retrofitting de CNC}, modernizando sistemas obsoletos com controladores e soluções de automação atuais.  
\item Concebi e executei projetos \textbf{mecânicos e elétricos}, incluindo esquemas elétricos utilizando \textbf{Eplan}.  
\item Integrei sistemas de \textbf{robótica} e \textbf{visão}.
\end{itemize}

\medskip

\cvevent{Estágio de Verão}{Volper}{Julho 2022 - Setembro 2022}{}
\begin{itemize}
\item Realizei o \textbf{retrofitting} de uma \textbf{máquina CNC}, integrando novos sistemas \textbf{elétricos}, drivers e controladores.  
\item Trabalhei com \textbf{sensores industriais} e programei \textbf{robôs} para automação de embalamento.  
\end{itemize}

\cvsection{Projetos}

\cvevent{Bolseiro de Investigação / Dissertação}{Laboratório de Automação e Robótica | Universidade de Aveiro}{Fevereiro 2023 - Junho 2023}{}
\begin{itemize}
\item Implementei um \textbf{algoritmo de variação da luz solar} num simulador baseado em dados.  
\item Integrei um avançado \textbf{algoritmo de variação de objetos} no simulador.  
\item O método proposto \textbf{scene augmentation} melhorou significativamente a capacidade dos modelos para generalizar dados nunca antes vistos.  
\end{itemize}

\medskip

\cvevent{Pytorch / ROS / Open3D}{Dora the Mug Finder}{Dezembro 2022 - Fevereiro 2023}{}
\begin{itemize}
\item Apliquei técnicas avançadas de \textbf{processamento de nuvem de pontos} para extração de objetos em cenários 3D.  
\item Utilizei uma \textbf{rede neural} para classificação de objetos em câmaras de cena.  
\item Desenvolvi com \textbf{ROS} para garantir colaboração e interoperabilidade eficientes.  
\end{itemize}

\medskip

\cvevent{Machine Learning / Engenharia Eletrotécnica}{Automec | Universidade de Aveiro}{Setembro 2021 - Presente}{}
\begin{itemize}
\item Desenvolvi \textbf{redes neurais} em \textbf{PyTorch} para controlo de veículos autónomos.  
\item Desenhei e construí uma \textbf{PCB} com distribuição de potência e controlo.  
\end{itemize}



\clearpage




%% If the NEXT page doesn't start with a \cvsection but you'd
%% still like to add a sidebar, then use this command on THIS
%% page to add it. The optional argument lets you pull up the 
%% sidebar a bit so that it looks aligned with the top of the
%% main column.
% \addnextpagesidebar[-1ex]{page3sidebar}

\end{document}
